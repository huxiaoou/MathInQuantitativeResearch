\section{模型假设}

一般的线性回归模型可以用下面的记号简单描述为

\begin{equation}
  \bm {Y} = \bm{X} \bm{b} + \bm{\epsilon}
\end{equation}

以及对样本给定的权重$\bm{w}$向量。其中,

\begin{itemize}
  \item $\bm{Y}$ 是$n\times 1$向量,即自变量,已知。
  \item $\bm{X}$ 是$n\times k$矩阵,即因变量,已知。一般地,应当要求$n>k$,且$\bm{X}$满秩,其秩为$k$。假设$\bm{X}$的某一列(不妨设为第一列)全部为1,则这是一个带有截距项的回归模型;否则(即没有任何一列全部为1),这是一个不带截距项的回归。
  \item $\bm{\epsilon}$ 是$n\times 1$向量,其每个元素$\epsilon_i$间相互独立,且$\E \epsilon_i=0, \Var(\epsilon_i)=\sigma_i^2$。注意,若不使用最大似然估计,可以不要求$\epsilon_i$服从正态分布。$\bm{\epsilon}$是未知不可观测的,只能通过回归对其估计。
  \item $\bm{w}$ 是 $n\times 1$向量,已知。若$\sigma_i = \sigma$,则所有样本等权回归,$w_i = \frac{1}{n}$; 否则,$w_i \propto \frac{1}{\sigma_i^2}$。
  \item $\bm{b}$是$p\times 1$向量,即各自变量的系数,待估计的对象。
\end{itemize}

