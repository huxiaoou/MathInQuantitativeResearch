\section{BARRA截面回归}

\subsection{符号与假设}

假设在固定时刻$t$,横截面上有$n$个股票,分属$p$个行业,以及对应有$q$个风险因子。$n$个股票的收益是$\bm{r}$,BARRA截面回归模型可以写为

\begin{equation}
    \bm{r} = \bm{Xf} + \bm{\epsilon}
\end{equation}

其中

\begin{itemize}
    \item $\bm{r}$是$N\times 1$向量,代表$N$只股票的收益率,已知。
    \item $\bm{X}$是$N\times K$的矩阵$(N>K)$,已知。$\bm{X}$形如
          \begin{equation}
              \bm{X} = \left[
                  \begin{array}{c ccc ccc}
                      1      & X_{1,I_1} & \cdots & X_{1, I_P} & X_{1,S_1} & \cdots & X_{1, S_Q} \\
                      1      & X_{2,I_1} & \cdots & X_{2, I_P} & X_{2,S_1} & \cdots & X_{2, S_Q} \\
                      \vdots & \vdots    & \vdots & \vdots     & \vdots    & \vdots & \vdots     \\
                      1      & X_{N,I_1} & \cdots & X_{N, I_P} & X_{N,S_1} & \cdots & X_{N, S_Q} \\
                  \end{array}
                  \right]
          \end{equation}
          $\bm{X}$按照列可以分成三部分如下:
          \begin{itemize}
              \item $\bm{X}$的第1列$\bm{X}_{\cdot1}$全部是1,其含义是代表市场(或者国家)暴露。自然地,该市场内所有股票都在该因子上有1个单位的暴露。
              \item $\bm{X}$的第2列到第$P+1$列是$N$个股票在$P$个行业上的暴露。即如果个股$n$属于行业$p$, 那么$\bm{X}_{n,I_P} = 1$,否则$\bm{X}_{n,I_P} = 0$。这里还要进一步要求假设对任何一个股票,属于且仅属于一个行业。依照这个定义,容易验证$\bm{X_{\cdot1}} = \bm{X_{\cdot I_1}} + \cdots + \bm{X_{\cdot I_P}}$。
              \item $\bm{X}$的最后$Q$列是$n$个股票在$Q$个一般因子上的暴露。对这些列仅要求其满秩,且与之前$1+P$列不共线即可。
          \end{itemize}
          从上述描述可以看出,$K=1+P+Q$,且$\bm{X}$的秩为$K-1=P+Q$。
    \item $\bm{f}$是$K\times 1$向量,代表$K$个因子的收益率,未知,是需要估计的对象。
    \item $\bm{\epsilon}$是$N\times 1$向量,代表$N$只股票的残差收益率,未知,是需要估计的对象。
    \item $\bm{w}$ 是$N\times 1$向量,已知,代表回归时每个样本的权重。若以$N\times 1$的向量$\bm{v}$代表$N$只股票的市值,通常假设
          \begin{equation}
              w_i \propto \sqrt{v_i}
          \end{equation}
          $\bm{w}$也可以根据经验直接给出。此外,以下还会经常使用以$\bm{w}$为对角元素的$N\times N$的对角矩阵$\bm{W}$。
\end{itemize}

\subsection{对共线性问题的处理}

由于$\bm{X}$本身不满秩,因此在上述假设下,问题

\begin{equation}\label{eq_barra_target}
    \min_{\bm{f}} \bm{(r - Xf)^TW(r - Xf)}
\end{equation}

的解不唯一。例如假设

\begin{equation}
    \bm{f}  =  (f_c, f_{I_1},  ..., f_{I_P}, f_{S_1},  ..., f_{S_Q})^T
\end{equation}

是问题\ref{eq_barra_target}的一个解,那么容易验证,给定$\delta \neq 0$

\begin{equation}
    \bm{\tilde{f}}  =  (f_c-\delta, f_{I_1}+\delta,  ..., f_{I_P}+\delta, f_{S_1},  ..., f_{S_Q})^T
\end{equation}

也是问题\ref{eq_barra_target}的一个解。为了解决这个问题,我们需要给出一个约束条件,让解唯一。该约束条件按照下面的推导给出,假设每个行业的市值为$s_{I_i}, 1 \leq  i \leq P$,且有

\begin{equation}
    \sum_{i=1}^Ps_{I_i} = 1
\end{equation}

对市场(国家)因子收益率$f_c$以及行业因子收益率$f_{I_i}$,从其定义出发,希望有

\begin{equation}\label{eq_barra_rel_fc_fp}
    \sum_{i=1}^P s_{I_i}(f_{I_i} + f_c) = f_c
\end{equation}

\ref{eq_barra_rel_fc_fp}式即表明对行业因子收益率$f_{I_i}$的真实含义:它是传统行业收益率相对市场(国家)因子收益率的超额收益,它与市场(国家)因子收益率相加之后再按照市值加权就是市场 (国家)因子收益率。\ref{eq_barra_rel_fc_fp}式化简后等效为

\begin{equation}
    \sum_{i=1}^P s_{I_i} f_{I_i} = 0
\end{equation}

有了这个约束条件,问题\ref{eq_barra_target}的解唯一了。

\subsection{问题的解}

\begin{equation}
    \left[
        \begin{array}{c}
            f_c     \\
            f_{I_1} \\
            \vdots  \\
            f_{I_P} \\
            f_{S_1} \\
            \vdots  \\
            f_{S_Q} \\
        \end{array}
        \right] =
    \left[
        \begin{array}{c ccc ccc}
            1      & 0                        & \cdots & 0                            & 0      & \cdots & 0      \\

            0      & 1                        & \cdots & 0                            & 0      & \cdots & 0      \\
            \vdots & \vdots                   & \vdots & \vdots                       & \vdots & \vdots & \vdots \\
            0      & 0                        & \cdots & 1                            & 0      & \cdots & 0      \\
            0      & -\frac{s_{I_1}}{s_{I_P}} & \cdots & -\frac{s_{I_{P-1}}}{s_{I_P}} & 0      & \cdots & 0      \\

            0      & 0                        & \cdots & 0                            & 1      & \cdots & 0      \\
            \vdots & \vdots                   & \vdots & \vdots                       & \vdots & \vdots & \vdots \\
            0      & 0                        & \cdots & 0                            & 0      & \cdots & 1      \\
        \end{array}
        \right]
    \left[
        \begin{array}{c}
            f_c         \\
            f_{I_1}     \\
            \vdots      \\
            f_{I_{P-1}} \\
            f_{S_1}     \\
            \vdots      \\
            f_{S_Q}     \\
        \end{array}
        \right]
\end{equation}

若记$\bm{f} = \bm{f_{K}}$,上式即

\begin{equation}
    \bm{f_{K}} = \bm{R}\bm{f_{K-1}}
\end{equation}

其中$\bm{R}$是$K\times (K-1)$的矩阵,通过它,将$\bm{f_K}$用$\bm{f_{K-1}}$线性表出。于是

\begin{equation}\label{eq_barra_lr_transform}
    \begin{aligned}
        \min_{\bm{f}} [\bm{(r - Xf)^TW(r - Xf)}] & = \min_{\bm{f_K}} [\bm{(r - Xf_K)^TW(r - Xf_K)}]                 \\
                                                 & = \min_{\bm{f_{K-1}}} [\bm{(r - XRf_{K-1})^TW(r - XRf_{K-1})}]   \\
                                                 & = \min_{\bm{f_{K-1}}} [\bm{(r - X^*f_{K-1})^TW(r - X^*f_{K-1})}] \\
    \end{aligned}
\end{equation}

其中

\begin{equation}
    \bm{X^*} = \bm{XR}
\end{equation}

此时\ref{eq_barra_lr_transform}式的最后一步是一个标准多元线性回归问题的求解,参考\ref{eq_lr_hat_b_sol}式,其解为

\begin{equation}
    \begin{array}{rcl}
        \bm{\hat{f}_{K-1}} & = & [\bm{X^{*T}WX^*}]^{-1}[\bm{X^{*T}Wr}] \\
                           & = & [\bm{R^TX^TWXR}]^{-1}[\bm{R^TX^TWr}]  \\
                           & = & \bm{Hr}                               \\
    \end{array}
\end{equation}

其中,$\bm{H}$是$(K-1)\times N$的矩阵,定义为

\begin{equation}
    \bm{H} = [\bm{R^TX^TWXR}]^{-1}\bm{R^TX^TW}
\end{equation}

容易看出,此时$\bm{f}$的解为

\begin{equation}
    \bm{\hat{f}} = \bm{\hat{f}_{K}} = \bm{R\hat{f}_{K-1}}=\bm{RHr}
\end{equation}

以及容易验证

\begin{equation}\label{eq_barra_HXR_is_1}
    \bm{HX^*} = \bm{HXR} = \bm{1_{K-1}}
\end{equation}

此外,若记

\begin{equation}
    \bm{\Omega} = \bm{RH}
\end{equation}

那么$\bm{\Omega X}$有如下形式

\begin{equation}\label{eq_barra_Omega_X}
    \bm{\Omega X} =
    \left[
        \begin{array}{rr rrrr rrr}
            1      & s_{I_1}   & s_{I_2}   & \cdots & s_{I_P}   & 0      & \cdots & 0      \\

            0      & 1-s_{I_1} & -s_{I_2}  & \cdots & -s_{I_P}  & 0      & \cdots & 0      \\
            0      & -s_{I_1}  & 1-s_{I_2} & \cdots & -s_{I_P}  & 0      & \cdots & 0      \\
            \vdots & \vdots    &           & \vdots & \vdots    & \vdots & \vdots & \vdots \\
            0      & -s_{I_1}  & -s_{I_2}  & \cdots & 1-s_{I_P} & 0      & \cdots & 0      \\

            0      & 0         &           & \cdots & 0         & 1      & \cdots & 0      \\
            \vdots & \vdots    &           & \vdots & \vdots    & \vdots & \vdots & \vdots \\
            0      & 0         &           & \cdots & 0         & 0      & \cdots & 1      \\
        \end{array}
        \right]
\end{equation}

这是因为\ref{eq_barra_Omega_X}式右边可以写为

\begin{equation}
    \ref{eq_barra_Omega_X}\text{式右边} = \bm{I_{K}} + \bm{US^T}
\end{equation}

其中$\bm{I_{(K)}}$表示$K$阶单位矩阵,而$\bm{U}$和$\bm{S}$的定义如下

\begin{equation}
    \bm{U} = \left[
        \begin{array}{c}
            1      \\
            -1     \\
            \vdots \\
            -1     \\
            0      \\
            \vdots \\
            0      \\
        \end{array}
        \right] \quad
    \bm{S} = \left[
        \begin{array}{c}
            0       \\
            s_{I_1} \\
            \vdots  \\
            s_{I_P} \\
            0       \\
            \vdots  \\
            0       \\
        \end{array}
        \right]
\end{equation}

从而要证明\ref{eq_barra_Omega_X}式等价于证明

\begin{equation}\label{eq_barra_Omega_X_alt}
    \bm{\Omega X - I_{K} - US^T} = \bm{0_{K\times K}}
\end{equation}

下面对这个结论给出证明。

\begin{proof}
    容易验证

    \begin{equation}
        \bm{S^TR} = \bm{0_{1\times(K-1)}}
    \end{equation}

    即向量$\bm{S}$与$\bm{R}$中的所有列向量正交。于是我们可以构造$\bm{A}$形如

    \begin{equation}
        \bm{A} = [\bm{R}, \bm{S}]
    \end{equation}

    注意$\bm{R}$是$K\times (K-1)$的满秩矩阵,$\bm{S}$是$K\times 1$的向量,且与$\bm{R}$中所有列向量正交,故而$\bm{A}$是一个$K\times K$的满秩矩阵。现在我们记\ref{eq_barra_Omega_X_alt}的左边为$\bm{Z}$,若能证明$\bm{ZA} = \bm{0}_{K\times K}$,则问题得证。而依照定义$\bm{ZA} = [\bm{ZR},\bm{ZS}]$,于是可以分两部来考察。

    \paragraph*{关于$\bm{ZR}$}
    容易验证
    \begin{equation}
        \bm{\Omega X R} = \bm{R [H X R]} = \bm{R I_{(K-1)\times(K-1)}} = \bm{R}
    \end{equation}

    于是
    \begin{equation}
        \bm{ZR} = \bm{\Omega X R} - \bm{I_{K}R} - \bm{U[S^TR]} = \bm{R-R-U0_{1\times (K-1)}} = \bm{0_{K\times(K-1)}}
    \end{equation}

    \paragraph*{关于$\bm{ZS}$}

    首先将$\bm{X}$按照列写为
    \begin{equation*}
        \bm{X} = [\bm{X_c}, \bm{X_1}, \cdots , \bm{X_P}, \bm{Y_1}, \cdots , \bm{Y_Q}]
    \end{equation*}

    这其中每个元素是$N\times 1$的向量,且依照模型定义
    \begin{equation}\label{eq_barra_sum_xi_is_xc}
        \bm{X_c}  = \bm{X_1} + \cdots + \bm{X_P}
    \end{equation}

    由于\ref{eq_barra_HXR_is_1}式,将其按照列展开即有
    \begin{equation}
        \bm{H}[\bm{X_c}, \bm{X_1}-\frac{s_{I_1}}{s_{I_P}}\bm{X_P}, \cdots, \bm{X_{P-1}}-\frac{s_{I_{P-1}}}{s_{I_P}}\bm{X_P}, \bm{Y_1}, \cdots, \bm{Y_Q}] = \bm{1_{K-1}}
    \end{equation}

    考察上式左右两边第$1$列有
    \begin{equation}\label{eq_barra_hxc}
        \bm{HX_c} = [1, \overbrace{0,\cdots, 0}^{P-1\text{个}}, \overbrace{0, \cdots, 0}^{Q\text{个}}]^T
    \end{equation}

    以及第$2$到$P$列有
    \begin{equation}\label{eq_barra_hxi}
        \bm{HX_i} =\frac{s_{I_i}}{s_{I_P}}\bm{HX_P} + [0, \overbrace{0,\cdots,1,\cdots, 0}^{\text{第}i+1\text{位是1,其余位是0}}, \overbrace{0, \cdots, 0}^{Q\text{个}}]^T \quad i=1,2,\cdots P-1
    \end{equation}

    在\ref{eq_barra_sum_xi_is_xc}式左右两边同时左乘$\bm{H}$,然后将$\ref{eq_barra_hxc}$式和$\ref{eq_barra_hxi}$代入有
    \begin{equation}
        \left[
            \begin{array}{c}
                1      \\
                0      \\
                \vdots \\
                0      \\
                0      \\
                \vdots \\
                0      \\
            \end{array}
            \right]_{1\times(1 + (P-1) + Q)} =
        \frac{\sum_{i=1}^Ps_{I_i}}{s_{I_P}}\bm{HX_P} +
        \left[
            \begin{array}{c}
                0      \\
                1      \\
                \vdots \\
                1      \\
                1      \\
                \vdots \\
                0      \\
            \end{array}
            \right]_{1\times(1 + (P-1) + Q)}
    \end{equation}

    若记$\bm{U_{K-1}}$为
    \begin{equation}
        \bm{U_{K-1}} = [1, \overbrace{-1,\cdots,-1}^{P-1\text{个}}, \overbrace{0, \cdots, 0}^{Q\text{个}}]^T
    \end{equation}

    注意到$\sum_{i=1}^Ps_{I_i}=1$, 那么此即
    \begin{equation}
        \frac{1}{s_{I_P}}\bm{HX_P} = \bm{U_{K-1}}
    \end{equation}

    现在来计算$\bm{HXS}$,同样将其按照$\bm{X}$的列展开,并结合\ref{eq_barra_hxi}有
    \begin{equation}\label{eq_barra_core}
        \bm{HXS} = \sum_{i=1}^Ps_{I_i}\bm{HX_i} =\frac{m}{s_{I_P}}\bm{HX_P} + \bm{S_{K-1}} = m\bm{U_{K-1}} + \bm{S_{K-1}}
    \end{equation}

    其中
    \begin{equation}
        m = \sum_{i=1}^Ps_{I_i}^2 = \bm{S^TS}
    \end{equation}

    以及
    \begin{equation}
        \bm{S_{K-1}} = [0, \overbrace{s_{I_1},\cdots,s_{I_{P-1}}}^{P-1\text{个}}, \overbrace{0, \cdots, 0}^{Q\text{个}}]^T
    \end{equation}

    并且,依照定义,容易验算有
    \begin{equation}\label{eq_barra_RU_RS}
        \left\{
            \begin{array}{rcl}
                \bm{RU_{K-1}} & = & \bm{U} + \frac{1}{s_{I_P}}\bm{\delta} \\
                \bm{RS_{K-1}} & = & \bm{S} - \frac{m}{s_{I_P}}\bm{\delta} \\
            \end{array}
        \right.
    \end{equation}

    其中
    \begin{equation}
        \bm{\delta} = [\overbrace{0, \dots, 0}^{P\text{个}},1,\overbrace{0, \dots, 0}^{Q\text{个}}]^T
    \end{equation}

    至此,在\ref{eq_barra_core}式两边左乘$\bm{R}$,并将\ref{eq_barra_RU_RS}式代入即有
    \begin{equation}
        \bm{RHXS} = m\bm{U} + \bm{S}
    \end{equation}
    
    此即
    \begin{equation}
        \bm{ZS} = \bm{0}_{K\times 1}
    \end{equation}

    于是问题得证。
\end{proof}

\begin{remark}
    $\bm{\Omega X}$的形式之所以如此重要,是因为它的实际意义,它代表了$K$个纯因子组合在$K$个因子上的暴露。
\end{remark}

