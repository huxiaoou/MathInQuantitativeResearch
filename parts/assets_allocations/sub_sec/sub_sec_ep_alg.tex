\subsection{数值优化模型}

熵池模型优化目标

\begin{equation}\label{eq_ep_opt_problem}
    \left\{
    \begin{aligned}
         & \min_{\vec{x} \geq 0}\vec{x}^T(\ln\vec{x}-\ln\vec{p}) \\
         & \vec{F}x \leq \vec{f}                                 \\
         & \vec{H}x = \vec{h}
    \end{aligned}
    \right.
\end{equation}

其中

\begin{itemize}
    \item $\vec{x}: J\times 1$ 向量,最终需要求解的对象,代表后验分布,也就是前文中的$\vec{\tilde{p}}$。
    \item $\vec{p}: J\times 1$ 向量,已知参数,代表先验分布。
    \item $\vec{F}: M\times J$ 矩阵,已知参数,代表不等关系约束类型的观点,每一行对应一个观点,一共$M$个观点,$M$可以为0,代表没有不等关系约束。
    \item $\vec{f}: M\times 1$ 矩阵,已知参数,代表不等关系约束的右侧,可以全部为零。
    \item $\vec{H}: N\times J$ 矩阵,已知参数,代表相等关系约束类型的观点,每一行对应一个观点,一共$N$个观点,一般情况下$N\geq1$,这是由于需要要求所有样本对应的概率之后为1,即$\vec{x}^T\vec{1}=1$,即$\vec{H}$至少有一个全部为1的行。
    \item $\vec{h}: M\times 1$ 矩阵,已知参数,代表相等关系约束的右侧,对应$H$全为1的那一行的元素是1。
\end{itemize}

使用Lagrange 乘子法,\ref{eq_ep_opt_problem}式的Lagrange形式是

\begin{equation}\label{eq_ep_opt_lagrange}
    \mathbf{L}(\vec{x}, \vec{\lambda}, \vec{\nu}) = \vec{x}^T(\ln\vec{x}-\ln\vec{p}) + \vec{\lambda}^T(\vec{F}\vec{x}-\vec{f}) + \vec{\nu}^T(\vec{H}\vec{x}-\vec{h})
\end{equation}

其中
\begin{itemize}
    \item $\vec{\lambda} = [\lambda_1,...,\lambda_M]^T$ 是 $M\times 1$向量
    \item $\vec{\nu} = [\nu_1,...,\nu_N]^T$ 是 $N\times 1$向量
\end{itemize}

\ref{eq_ep_opt_lagrange}式对$\vec{x}$求偏导有

\begin{equation}
    \frac{\partial \mathbf{L}}{\partial \vec{x}} = \ln\vec{x}-\ln\vec{p} + \vec{1}_{J\times 1} + \vec{F}^T\vec{\lambda} + \vec{H}^T\vec{\nu}
\end{equation}

由此可知

\begin{equation}\label{eq_x_lbd_nu}
    \vec{x}(\vec{\lambda},\vec{\nu}) = \exp\Big\{\ln\vec{p} - \vec{1}_{J\times 1} - \vec{F}^T\vec{\lambda} - \vec{H}^T\vec{\nu}\Big\}
\end{equation}

定义$\mathbf{L}$的共轭函数$\mathbf{G}$为

\begin{equation}
    \mathbf{G}(\vec{\lambda},\vec{\nu}) = \mathbf{L}(\vec{x}(\vec{\lambda}, \vec{\nu}), \vec{\lambda}, \vec{\nu})
\end{equation}

参考Lagrange乘子法理论,原问题等效于

\begin{equation}\label{eq_ep_opt_dual}
    (\vec{\lambda}^*,\vec{\nu}^*) = \argmin_{\vec{\lambda}\geq \vec{0}, \vec{\nu}} \{\mathbf{G}(\vec{\lambda},\vec{\nu})\}
\end{equation}

因此,最终直接优化$\mathbf{G}$即可。

上述推导将问题从优化$\mathbf{L}$变形为优化$\mathbf{G}$,需要优化的变量个数由$J$调整为$M+N$。在实际计算中,$J$(抽样数量)通常很大,而$M,N$(观点数量)通常相对较小,这样就大大简化了问题难度。同时$\mathbf{G}$关于$(\vec{\lambda}, \vec{\nu})$的一阶偏导(Jacob向量)可以直接给出(在下一节给出),这样也可以大大加速优化过程。
