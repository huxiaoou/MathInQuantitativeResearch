\subsection{模型概述}

假设有$P$个风险因子,服从先验联合分布$U$,从$U$中抽取$J$个样本形成样本集

\begin{equation}
    \vec{X} = \left[
        \begin{aligned}
            X_{1,1} & \quad ...    & \quad X_{1,P} \\
            \vdots  & \quad \vdots & \quad \vdots  \\
            X_{J,1} & \quad ...    & \quad X_{J,P} \\
        \end{aligned}
        \right]
\end{equation}

以及对应的先验分布$\vec{p}$

\begin{equation}
    \vec{p} = [\frac{1}{J},...,\frac{1}{J}]^T
\end{equation}

一般情况下,$\vec{p}$是均匀的,即每一个样本的概率都是$\frac{1}{J}$,特定情况下可以根据问题的具体形态做适当调整。

假设投资者有$Q$个观点,这些观点可以写成$P$个风险因子的函数,即

\begin{equation}
    \left\{
    \begin{aligned}
        g_1 & = g_1(x_1,...,x_P) \\
        g_2 & = g_2(x_1,...,x_P) \\
        ... & ...                \\
        g_Q & = g_Q(x_1,...,x_P) \\
    \end{aligned}
    \right.
\end{equation}

对每一种先验分布运用观点,可以得到一个$J\times Q$的观点样本矩阵

\begin{equation}
    \vec{V} = \left[
        \begin{aligned}
            g_{1,1} & \quad ...    & \quad g_{1,Q} \\
            \vdots  & \quad \vdots & \quad \vdots  \\
            g_{J,1} & \quad ...    & \quad g_{J,Q} \\
        \end{aligned}
        \right]
\end{equation}

$\vec{V}$的先验分布也为$\vec{p}$,假设投资者关心的是后验分布$\vec{\tilde{p}}$,对观点的约束整理为不等式类和等式类,即

\begin{equation}
    \vec{V} = [\vec{F}_{M\times J}, \vec{H}_{N\times J}]
\end{equation}

不等式约束为
\begin{equation}\label{eq_cons_Ff}
    \vec{F}\vec{\tilde{p}} \leq \vec{f}
\end{equation}

等式约束为
\begin{equation}\label{eq_cons_Hf}
    \vec{H}\vec{\tilde{p}} = \vec{h}
\end{equation}

其中$M+N=Q$,模型的最终目标是找到这样的后验$\vec{\tilde{p}}$,该后验分布满足
\begin{itemize}
    \item 与先验分布的熵距离尽可能小,即$\vec{\tilde{p}}^T(\ln\vec{\tilde{p}}-\ln\vec{p})$尽可能小。
    \item 满足约束\ref{eq_cons_Ff}式和\ref{eq_cons_Hf}式
\end{itemize}

\subsubsection{一个例子}
假设三个资产的收益率(三项风险因子)$\vec{x}=(x_1,x_2,x_3)$ 满足多元正态分布

\begin{equation}
    \vec{x} \sim \mathbf{N}(\vec{\mu}_{3\times1}, \Sigma_{3\times3})
\end{equation}

从这个分布中抽取$J=10000$个样本形成

\begin{equation}
    \vec{X} = \left[
        \begin{aligned}
            x_{1,1}, & \quad x_{1,2}, & \quad x_{1,3} \\
            \vdots   & \quad \vdots   & \quad \vdots  \\
            x_{J,1}, & \quad x_{J,2}, & \quad x_{J,3} \\
        \end{aligned}
        \right]
\end{equation}

假设对未来的观点是

\begin{itemize}
    \item 资产1收益的平方不高于资产2收益再加5个百分点,即$x_1^2 - x_2 \leq 0.05$
    \item 资产2收益不低于资产3收益再加3个百分点,即$x_2 - x_3 \geq 0.03$
    \item 资产3预期收益为4\%,即$x_3 = 0.04$
\end{itemize}

那么观点矩阵可以写为

\begin{equation}
    \vec{V} = \left[
        \begin{aligned}
            x_{1,1}^2 - x_{1,2}, & \quad x_{1,3} - x_{1,2}, & \quad x_{1,3}, & \quad 1      \\
            \vdots               & \quad \vdots             & \quad \vdots   & \quad \vdots \\
            x_{J,1}^2 - x_{J,2}, & \quad x_{J,3} - x_{J,2}, & \quad x_{J,3}, & \quad 1      \\
        \end{aligned}
        \right]=   [\vec{F}, \vec{H}]
\end{equation}

其中
\begin{equation}
    \vec{F} = \left[
        \begin{aligned}
            x_{1,1}^2 - x_{1,2}, & \quad x_{1,3} - x_{1,2} \\
            \vdots               & \quad \vdots            \\
            x_{J,1}^2 - x_{J,2}, & \quad x_{J,3} - x_{J,2} \\
        \end{aligned}
        \right]
    \quad
    \vec{H} = \left[
        \begin{aligned}
            \quad x_{1,3}, & \quad 1      \\
            \quad \vdots   & \quad \vdots \\
            \quad x_{J,3}, & \quad 1      \\
        \end{aligned}
        \right]
\end{equation}

而约束条件可以写为

\begin{equation}
    \vec{F}^T\vec{\tilde{p}} \leq \vec{f} = \left[
        \begin{aligned}
            0.05 \\
            -0.03
        \end{aligned}
        \right]
    \quad
    \vec{H}^T\vec{\tilde{p}} = \vec{h} = \left[
        \begin{aligned}
            0.04 \\
            1
        \end{aligned}
        \right]
\end{equation}

其中$\vec{H}$中比前述的观点中多增加了一个全为1的列是为了约束后验分布之和为1。

\subsubsection{对模型的补充说明}

\begin{enumerate}
    \item 风险因子可以不是收益率,但是其先验分布作为模型的输入,在实际运用中应当是已知或者能够计算的。
    \item “观点”是风险因子的函数,函数本身可以非线性。
    \item $\vec{V}$的每一行对应一次观察(或者说抽样),是各个观点在对应的风险因子下的实现。
    \item $\vec{V}$的每一列对应一个观点,是该观点在各次抽样下的实现。
\end{enumerate}